\documentclass{ctexart}
\usepackage{amsmath}
\usepackage{array}
\usepackage{tabularx}
\usepackage{longtable}
\usepackage{hyperref}
\usepackage{CJKfntef}

\begin{document}

\title{KPI V0.3.0}
\author{saitewasreset}

\maketitle

\tableofcontents

\section{主约束条件}

\begin{description}
    \item[$A_1$] 故意友伤是不可接受的.
          \begin{description}
              \item[$A_{11}$] 友伤使队友死亡再救起是严重“亏损”的.
          \end{description}
    \item[$A_2$] 因“奋战”而倒地是可以接受的.
    \item[$A_3$] 因“奋战”而吃补给是可以接受的.
    \item[$A_4$] 不建议“极限一换一”.
    \item[$A_5$] 因“奋战”而不小心友伤是可以接受的.
\end{description}

\section{符号表}


\begin{longtable}{|>{\centering\arraybackslash}p{3em}|>{\centering\arraybackslash}p{30em}|}
    \hline

    符号         & 含义                                                                                                                 \\

    \hline

    $D$        & 输出:对敌人的实际伤害\footnotemark.\textbf{不含造成的友伤.}                                                                         \\

    \hline

    $H$        & 友伤:造成的实际友伤.                                                                                                        \\

    \hline

    $D_A$      & 全输出:$D_A = D + H$.                                                                                                 \\

    \hline

    $k$        & 击杀数.                                                                                                               \\

    \hline

    $k^\alpha$ & 带权击杀数:$k^\alpha = \sum_{i = 1}^{n} k_i \cdot p^\alpha_i$,其中$k_i$为第$i$种敌人的击杀数,$p^\alpha_i$为在权值表$\alpha$下第$i$种敌人的权重. \\

    \hline

    $D^\alpha$ & 带权输出:$D^\alpha = \sum_{i = 1}^{n} D_i \cdot p^\alpha_i$,其中$D_i$为第$i$种敌人的击杀数,$p^\alpha_i$为在权值表$\alpha$下第$i$种敌人的权重.  \\

    \hline

    $n$        & 硝石采集量.                                                                                                             \\

    \hline

    $m^\alpha$        & 带权矿物采集量(含硝石):$m^\alpha = \sum_{i = 1}^{n} m_i \cdot p^\alpha_i$,其中$m_i$为第$i$种矿物的采集量,$p^\alpha_i$为在权值表$\alpha$下第$i$种矿物的权重.                                                                                                    \\

    \hline

    $s$        & 使用补给次数.                                                                                                            \\

    \hline

    $\bar{s}$  & 约化补给次数$\bar{s} =s + 1$.                                                                                            \\

    \hline

    $p$        & 玩家指数:$p = \frac{T_{p}}{T}$,其中$T_{p}$为该玩家处于该任务中的时间,$T$为任务总时间.                                                       \\

    \hline
\end{longtable}


\footnotetext{由伤害前后敌人血量变化量表示,不含护甲破坏.}

\section{指数定义}

定义角色代号:钻机——$D$,枪手——$G$,工程——$E$,(辅助型)侦察——$S$,输出型侦察——$S'$.

\subsection{人数及角色修正因子}

由于一局游戏的人数不同、角色分配不同,击杀数$k$、输出$D$、硝石采集量$n$、资源采集量$m$的分布可能有较大差异,若直接利用玩家$i$的数据$k_i$、$D_i$、$n_i$、$m_i$占总数据的比例作为相应指数的值,则计算出的KPI结果可能随局内玩家人数不同、角色分配不同而存在较大差异,不利于保持KPI的稳定性及参考性.
故需要引入修正因子$\Gamma^i, i \in \{k, D, n, m\}$对相关数据进行修正.

定义修正指标$\gamma_k^i$,其中$k$表示玩家$k$所选角色$i$对应的修正指标,修正指标数据见表\ref{tab:kill_by_character}、表\ref{tab:damage_by_character}、表\ref{tab:nitra_by_character}、表\ref{tab:minerals_by_character}.

对每一局游戏,对每种需要修正的数据$i, i \in \{k, D, n, m\}$,定义修正指标和$\delta^i$为:$\sum_{k = 1}^{n} \gamma_k^i$.

取附录\ref{sec:statistic}中的修正指标和为“标准”指标和$\delta_0^i$.

由此可定义修正因子为$\Gamma^i = \frac{\delta^i}{\delta_0^i}, i \in \{k, D, n, m\}$.

例如:

若游戏人数为4,且所选角色为$[D, G, E, S]$,对于击杀数$k$,由表\ref{tab:kill_by_character},修正指标和$\delta^k = 1.682 + 1.682 + 2.848 + 1.000 = 7.212$,“标准”指标和$\delta_0^k = 7.212$,修正因子为$\Gamma^i = \frac{7.212}{7.212} = 1.000$.

若游戏人数为4,且所选角色为$[E, E, E, E]$,对于击杀数$k$,由表\ref{tab:kill_by_character},修正指标和$\delta^k = 2.848 + 2.848 + 2.848 + 2.848 = 11.392$,“标准”指标和$\delta_0^k = 7.212$,修正因子为$\Gamma^i = \frac{11.392}{7.212} = 1.580$.

\subsection{友伤指数算法}
\label{sec:f}

定义$f(x) = \frac{99}{x - 1} + 100$,其中$x$为友伤比例$=\frac{H_i}{D_{A_i}}$.($A_1$)($A_5$)

$f(x)$的定义域为$[0, 1]$,值域为$(-\infty, 1]$.


\begin{longtable}{|>{\centering\arraybackslash}p{2em}|>{\centering\arraybackslash}p{4em}|}

    \hline

    $x \cdot 10^{-3}$ & $f(x)$    \endhead

    \hline

    $ 0.0 $           & $1.000$            \\

    \hline

    $ 1.5 $           & $0.851$            \\

    \hline

    $ 2.5 $           & $0.752$            \\

    \hline

    $ 3.5 $           & $0.652$            \\

    \hline

    $ 4.0 $           & $0.602$            \\

    \hline

    $ 6.5 $           & $0.352$            \\

    \hline

    $ 10 $            & $0.000$            \\

    \hline

    $ 15 $            & $-0.508$           \\

    \hline

    $ 20 $            & $-1.020$           \\

    \hline

    $ 100 $           & $-10.00$           \\

    \hline

    $ 1000 $          & $-\infty$          \\

    \hline
\end{longtable}

\subsection{指数定义}
\label{sec:idxdef}

\begin{longtable}{|>{\centering\arraybackslash}p{4em}|>{\centering\arraybackslash}p{14em}|>{\centering\arraybackslash}p{10em}|>{\centering\arraybackslash}p{4em}|}

    \hline

    指数符号         & 指数释义                                                & 定义                                                                        & 值域             \endhead

    \hline

    $k_I^\alpha$ & 击杀数指数:击杀数占总击杀数的比例(带权值,采用权值表$\alpha$,带修正因子).         & $min\{1, \frac{k_x^\alpha}{\sum_{i = 1}^{n} k_i^\alpha} \cdot \Gamma^k\}$ & $[0, 1]$                \\

    \hline

    $D_I^\alpha$ & 输出指数:输出占总输出的比例(带权值,采用权值表$\alpha$,带修正因子).            & $min\{1, \frac{D_x^\alpha}{\sum_{i = 1}^{n} D_i^\alpha} \cdot \Gamma^D\}$ & $[0, 1]$                \\

    \hline

    $P_I$        & 高威胁目标:使用高威胁权值表的输出指数.                                & $\frac{D_x^\delta}{\sum_{i = 1}^{n} D_i^\delta}$                          & $[0, 1]$                \\

    \hline

    $r_I$        & 救人指数:救人次数占总救人次数的比例,\textbf{若总救人次数为0,则为1}.           & $\frac{r_x}{\sum_{i = 1}^{n} r_i}$                                        & $[0, 1]$                \\

    \hline

    $d_I$        & 倒地指数:倒地次数占总倒地次数的比例,\textbf{若总倒地次数为0,则为0}.           & -$\frac{d_x}{\sum_{i = 1}^{n} d_i}$                                       & $[-1, 0]$               \\

    \hline

    $f_I$        & 友伤指数:见$\ref{sec:f}$节.                               & 见$\ref{sec:f}$节.                                                          & $(-\infty, 1]$          \\

    \hline

    $n_I$        & 硝石指数:采集硝石量占总硝石采集量的比例(带修正因子),\textbf{若总硝石采集量为0,则为0}. & $min\{1, \frac{n_x^\alpha}{\sum_{i = 1}^{n} n_i^\alpha} \cdot \Gamma^n\}$ & $[0, 1]$                \\

    \hline

    $m_I$        & 采集指数:采集矿石量占总矿石采集量的比例(带修正因子),\textbf{若总矿石采集量为0,则为0}. & $min\{1, \frac{m_x^\alpha}{\sum_{i = 1}^{n} m_i^\alpha} \cdot \Gamma^m\}$ & $[0, 1]$                \\

    \hline

    $s_I$        & 补给指数:补给次数占总补给次数的比例,\textbf{若总补给次数为0,则为0}.           & $-\frac{s_x^\alpha}{\sum_{i = 1}^{n} s_i^\alpha}$                         & $[-1, 0]$               \\

    \hline
\end{longtable}


\subsection{赋分算法}

由于不同角色完成相同任务的难度不同\footnote{例如,对于工程,带权输出占到该局游戏总计带权输出的50\%较为困难,而对于侦察,采集量占到该局总计采集量的50\%相对容易.},需要对直接计算出的指数进行赋分,再参加原始KPI计算.

\subsubsection{赋分原则}

对于由修正因子修正后的指数$I$,计算其在所有玩家中的排名$r$,由排名确定赋分区域$A$,每个赋分区域含以下信息:排名区间、原始分区间、赋分区间.

设赋分区间$A$中原始分(即$I$)的最小值为$I_{m}^s$,最大值为$I_{M}^s$,赋分的最小值为$I_{m}^t$,最大值为$I_{M}^t$,玩家原始分为$I^s$,则按如下公式计算得到赋分$I^t$:

\begin{equation}
    \frac{I^s - I_{m}^s}{I_{M}^s - I_{m}^s} = \frac{I^t - I_{m}^t}{I_{M}^t - I_{m}^t}
\end{equation}

\subsubsection{赋分实例}

为了简便起见,以下计算采用不带权、不带修正因子的指数$I$.

例如,对于枪手,其输出指数$D$的赋分区域信息如下:

\begin{longtable}{|>{\centering\arraybackslash}p{6em}|>{\centering\arraybackslash}p{6em}|>{\centering\arraybackslash}p{6em}|}
    \hline

    排名区间             & 原始分区间  & 赋分区间     \endhead

    \hline

    $[0\%, 15\%)$ & $[0.33, 1.00)$  & $[0.90, 1.00)$          \\

    \hline

    $[15\%, 40\%)$  & $[0.26, 0.32)$ & $[0.70, 0.90)$          \\

    \hline

    $[40\%, 50\%)$   & $[0.23, 0.25)$ & $[0.60, 0.70)$          \\

    \hline

    $[50\%, 70\%)$    & $[0.19, 0.23)$ & $[0.35, 0.6)$          \\

    \hline

    $[70\%, 100\%)$    & $[0.00, 0.19)$ & $[0.00, 0.35)$          \\

    \hline

\end{longtable}

若某枪手在某局游戏中的$D_I^s = 0.26$,其排名为$37\%$,由上表可知其赋分区间为$[15\%, 40\%)$,计算可得赋分为$D_I^t = 0.72$.

\subsubsection{赋分特性}

对于枪手,其采集指数$m$的赋分区域信息如下:

\begin{longtable}{|>{\centering\arraybackslash}p{6em}|>{\centering\arraybackslash}p{6em}|>{\centering\arraybackslash}p{6em}|}
    \hline

    排名区间             & 原始分区间  & 赋分区间     \endhead

    \hline

    $[0\%, 15\%)$ & $[0.22, 1.00)$  & $[0.90, 1.00)$          \\

    \hline

    $[15\%, 40\%)$  & $[0.15, 0.21)$ & $[0.70, 0.90)$          \\

    \hline

    $[40\%, 50\%)$   & $[0.12, 0.14)$ & $[0.60, 0.70)$          \\

    \hline

    $[50\%, 70\%)$    & $[0.06, 0.12)$ & $[0.35, 0.6)$          \\

    \hline

    $[70\%, 100\%)$    & $[0.00, 0.06)$ & $[0.00, 0.35)$          \\

    \hline

\end{longtable}

若某枪手在某局游戏中的$m_I^s = 0.19$,其排名为$23\%$,由上表可知其赋分区间为$[15\%, 40\%)$,计算可得赋分为$m_I^t = 0.82$.

若某枪手在某局游戏中的$m_I^s = 0.24$,其排名为$13\%$,由上表可知其赋分区间为$[0\%, 15\%)$,计算可得赋分为$m_I^t = 0.90$.

若某枪手在某局游戏中的$m_I^s = 0.47$,其排名为$1\%$,由上表可知其赋分区间为$[0\%, 15\%)$,计算可得赋分为$m_I^t = 0.93$.

对于侦察,其采集指数$m$的赋分区域信息如下:

\begin{longtable}{|>{\centering\arraybackslash}p{6em}|>{\centering\arraybackslash}p{6em}|>{\centering\arraybackslash}p{6em}|}
    \hline

    排名区间             & 原始分区间  & 赋分区间     \endhead

    \hline

    $[0\%, 15\%)$ & $[0.38, 1.00)$  & $[0.90, 1.00)$          \\

    \hline

    $[15\%, 40\%)$  & $[0.27, 0.37)$ & $[0.70, 0.90)$          \\

    \hline

    $[40\%, 50\%)$   & $[0.20, 0.25)$ & $[0.60, 0.70)$          \\

    \hline

    $[50\%, 70\%)$    & $[0.15, 0.19)$ & $[0.35, 0.6)$          \\

    \hline

    $[70\%, 100\%)$    & $[0.00, 0.13)$ & $[0.00, 0.35)$          \\

    \hline

\end{longtable}

若某侦察在某局游戏中的$m_I^s = 0.19$,其排名为$54\%$,由上表可知其赋分区间为$[50\%, 70\%)$,计算可得赋分为$m_I^t = 0.59$.

若某侦察在某局游戏中的$m_I^s = 0.24$,其排名为$43\%$,由上表可知其赋分区间为$[40\%, 50\%)$,计算可得赋分为$m_I^t = 0.67$.

若某侦察在某局游戏中的$m_I^s = 0.47$,其排名为$9\%$,由上表可知其赋分区间为$[0\%, 15\%)$,计算可得赋分为$m_I^t = 0.91$.



\textbf{由上述实例不难发现,在选择的角色\CJKunderdot{本职工作}上投入更多精力,获得的赋分提升更大。}

\section{角色任务KPI}
\label{sec:rKPI}

令$\beta_i \in \{k_I, D_I, P_I, r_I, d_I, f_I, n_I, m_I, s_I\}$为对应指数的权重,且$\beta_i$对应指数的值域的上界为$t_i$,
则“最大加权和”为$\sum_{i = 1}^{n} \beta_i \cdot t_i$,
设玩家本局中指数$\beta_i$对应的值为$a_i^t$(若该项目赋分,则为赋分后的值),则实际加权和为$\sum_{i = 1}^{n} \beta_i \cdot a_i^t$.

则定义任务KPI为:$mKPI = \frac{\sum_{i = 1}^{n} \beta_i \cdot a_i^t}{\sum_{i = 1}^{n} \beta_i \cdot t_i} \cdot 100$.

\subsection{钻机}

\begin{description}
    \item[$D_1$] 对群\cite{tieba-all}.
    \item[$D_2$] 弱远程.
\end{description}

约束条件:

\begin{description}
    \item[$D_1$] $k_D > D_D, k_D + D_D + P_D \ge 0.5$.
    \item[$D_2$] $P_I = 0$.
    \item[$A_1$] $f_D \ge 0.1$.
    \item[$A_2$]  我们估计,平均每局总计输出大约为70K,钻机平均输出应大于10K,则不带权输出指数$D_I$应大约为$D^0 = \frac{1}{7}$,而带权输出指数$D^D_I$应大于$D_I$. 我们估计,钻机每局\textbf{正常}倒地次数为2次($d^0$),每局所有人总计倒地次数为6次($d^1$),则由$A_2$:

          $\frac{D^0}{d^0} \cdot D_D> \frac{1}{d^1} \cdot d_D$
          解得$D_D > \frac{d^0 \cdot d_D}{d^1 \cdot D^0} = \frac{7}{3} \cdot d_D$.
    \item[$A_3$] 我们估计,钻机的补给指数约为$\frac{1}{4}$($s^0$)(钻机$\frac{1}{4}$,枪手$\frac{1}{4}$,工程$\frac{3}{8}$,侦察$\frac{1}{8}$)

          我们估计,5K伤害对应一份补给,即一份补给对应约$\frac{D^0}{2} = \frac{1}{14}$

          则由$A_3$:$\frac{D^0}{2} \cdot D_D > s^0 \cdot s_D$

          解得$s_D < \frac{D^0 \cdot D_D}{2 s^0} = \frac{2}{7} D_D$.

    \item[$A_4$] $d_D \ge r_D$.
\end{description}


\begin{longtable}{|>{\centering\arraybackslash}p{8em}|>{\centering\arraybackslash}p{4em}|>{\centering\arraybackslash}p{6em}|}
    \hline

    项目             & 权重标识  & 参考值     \endhead

    \hline

    击杀数指数($k_I^D$) & $k_D$ & $0.400$          \\

    \hline

    输出指数($D_I^D$)  & $D_D$ & $0.200$          \\

    \hline

    高威胁目标($P_I$)   & $P_D$ & $0.000$          \\

    \hline

    救人指数($r_I$)    & $r_D$ & $0.080$          \\

    \hline

    倒地指数($d_I$)    & $d_D$ & $0.085$          \\

    \hline

    友伤指数($f_I$)    & $f_D$ & $0.100$          \\

    \hline

    硝石指数($n_I$)    & $n_D$ & $0.048$          \\

    \hline

    补给指数($s_I$)    & $s_D$ & $0.057$          \\

    \hline

    采集指数($m_I$)    & $m_D$ & $0.030$          \\

    \hline
\end{longtable}


\subsection{枪手}

\begin{description}
    \item[$G_1$] 提供强有力的火力支援\cite{tieba-all}\cite{xiaoheihe-all} 对单+对群\cite{tieba-all}.
\end{description}

约束条件:

\begin{description}
    \item[$G_1$] $k_G + D_G + P_G \ge 0.75, D_G > P_G > k_G$.
    \item[$A_1$] $f_G \ge 0.1$.
    \item[$A_2$]  我们估计,平均每局总计输出大约为70K,枪手平均输出应大于20K,则不带权输出指数$D_I$应大约为$D^0 = \frac{2}{7}$,而带权输出指数$D^G_I$应大于$D_I$

          我们估计,枪手每局\textbf{正常}倒地次数为2次($d^0$),每局所有人总计倒地次数为6次($d^1$),则由$A_2$:

          $\frac{D^0}{d^0} \cdot D_G> \frac{1}{d^1} \cdot d_G$
          解得$D_G > \frac{d^0 \cdot d_G}{d^1 \cdot D^0} = \frac{7}{6} \cdot d_G$
    \item[$A_3$] 我们估计,枪手的补给指数约为$\frac{1}{4}$($s^0$)(钻机$\frac{1}{4}$,枪手$\frac{1}{4}$,工程$\frac{3}{8}$,侦察$\frac{1}{8}$)

          我们估计,8K伤害对应一份补给,即一份补给对应约$\frac{2D^0}{5} = \frac{4}{35}$

          则由$A_3$:$\frac{2D^0}{5} \cdot D_G > s^0 \cdot s_G$

          解得$s_G < \frac{2D^0 \cdot D_G}{5 s^0} = \frac{16}{35} D_G$

    \item[$A_4$] $d_G \ge r_G$.
\end{description}


\begin{longtable}{|>{\centering\arraybackslash}p{8em}|>{\centering\arraybackslash}p{4em}|>{\centering\arraybackslash}p{6em}|}
    \hline

    项目             & 权重标识  & 参考值     \endhead

    \hline

    击杀数指数($k_I^G$) & $k_G$ & $0.050$          \\

    \hline

    输出指数($D_I^G$)  & $D_G$ & $0.410$          \\

    \hline

    高威胁目标($P_I$)   & $P_G$ & $0.300$          \\

    \hline

    救人指数($r_I$)    & $r_G$ & $0.050$          \\

    \hline

    倒地指数($d_I$)    & $d_G$ & $0.050$          \\

    \hline

    友伤指数($f_I$)    & $f_G$ & $0.100$          \\

    \hline

    硝石指数($n_I$)    & $n_G$ & $0.010$          \\

    \hline

    补给指数($s_I$)    & $s_G$ & $0.025$          \\

    \hline

    采集指数($m_I$)    & $m_G$ & $0.005$          \\

    \hline
\end{longtable}


\subsection{工程}

\begin{description}
    \item[$E_1$] 输出\cite{tieba-all}\cite{xiaoheihe-all}.
\end{description}

约束条件:

\begin{description}
    \item[$E_1$] $k_E + D_E + P_E \ge 0.65, D_E > P_E > k_E$.
    \item[$A_1$] $f_E \ge 0.1$.
    \item[$A_2$]  我们估计,平均每局总计输出大约为70K,工程平均输出应大于30K,则不带权输出指数$D_I$应大约为$D^0 = \frac{3}{7}$,而带权输出指数$D^E_I$应大于$D_I$

          我们估计,工程每局\textbf{正常}倒地次数为2次($d^0$),每局所有人总计倒地次数为6次($d^1$),则由$A_2$:

          $\frac{D^0}{d^0} \cdot D_E> \frac{1}{d^1} \cdot d_E$
          解得$D_E > \frac{d^0 \cdot d_E}{d^1 \cdot D^0} = \frac{7}{9} \cdot d_E$
    \item[$A_3$] 我们估计,工程的补给指数约为$\frac{3}{8}$($s^0$)(钻机$\frac{1}{4}$,枪手$\frac{1}{4}$,工程$\frac{3}{8}$,侦察$\frac{1}{8}$)

          我们估计,10K伤害对应一份补给,即一份补给对应约$\frac{D^0}{3} = \frac{1}{7}$

          则由$A_3$:$\frac{D^0}{3} \cdot D_E > s^0 \cdot s_E$

          解得$s_E < \frac{D^0 \cdot D_E}{3 s^0} = \frac{8}{21} D_E$

    \item[$A_4$] $d_E \ge r_E$.
\end{description}


\begin{longtable}{|>{\centering\arraybackslash}p{8em}|>{\centering\arraybackslash}p{4em}|>{\centering\arraybackslash}p{6em}|}
    \hline

    项目             & 权重标识  & 参考值     \endhead

    \hline

    击杀数指数($k_I^E$) & $k_E$ & $0.050$          \\

    \hline

    输出指数($D_I^E$)  & $D_E$ & $0.500$          \\

    \hline

    高威胁目标($P_I$)   & $P_E$ & $0.125$          \\

    \hline

    救人指数($r_I$)    & $r_E$ & $0.050$          \\

    \hline

    倒地指数($d_I$)    & $d_E$ & $0.050$          \\

    \hline

    友伤指数($f_I$)    & $f_E$ & $0.100$          \\

    \hline

    硝石指数($n_I$)    & $n_E$ & $0.025$          \\

    \hline

    补给指数($s_I$)    & $s_E$ & $0.080$          \\

    \hline

    采集指数($m_I$)    & $m_E$ & $0.020$          \\

    \hline
\end{longtable}


\subsection{(辅助型)侦察}

\begin{description}
    \item[$S_1$] 保证硝石供应.\cite{bilibili-scout}\cite{tieba-scout}\cite{tieba-all}\cite{xiaoheihe-all}
    \item[$S_2$] 处理高威胁单位.\cite{bilibili-scout}\cite{tieba-scout}\cite{tieba-all}
    \item[$S_3$] 杀敌不是主要工作.\cite{tieba-scout}\cite{tieba-all}\cite{xiaoheihe-all}
    \item[$S_4$] 采矿.\cite{tieba-all}\cite{xiaoheihe-all}
\end{description}

约束条件:

\begin{description}
    \item[$S_1$] $n_S \ge 0.3$.
    \item[$S_2$] $P_S > D_S, P_S > k_S, P_S \ge 0.1$.
    \item[$S_3$] $k_S + D_S + P_S \le 0.2$.
    \item[$S_4$] $n_S + p_S \ge 0.5$.
    \item[$A_1$] $f_E \ge 0.1$.
    \item[$A_2$] 不适用.\cite{tieba-scout}\cite{tieba-all}\cite{xiaoheihe-all}
    \item[$A_3$] 我们估计,侦察的补给指数约为$\frac{1}{8}$($s^0$)(钻机$\frac{1}{4}$,枪手$\frac{1}{4}$,工程$\frac{3}{8}$,侦察$\frac{1}{8}$),$D^0 = \frac{4}{35}$

          我们估计,4K伤害对应一份补给(不然灯真的不够用啊\verb|QAQ|),即一份补给对应约$\frac{D^0}{2} = \frac{2}{35}$

          则由$A_3$:$\frac{D^0}{2} \cdot D_S > s^0 \cdot s_S$

          解得$s_S < \frac{D^0 \cdot D_S}{2 s^0} = \frac{16}{35} D_S$.

    \item[$A_4$] $d_S \ge r_S$.
\end{description}


\begin{longtable}{|>{\centering\arraybackslash}p{8em}|>{\centering\arraybackslash}p{4em}|>{\centering\arraybackslash}p{6em}|}
    \hline

    项目             & 权重标识  & 参考值     \endhead

    \hline

    击杀数指数($k_I^S$) & $k_S$ & $0.010$          \\

    \hline

    输出指数($D_I^S$)  & $D_S$ & $0.040$          \\

    \hline

    高威胁目标($P_I$)   & $P_S$ & $0.150$          \\

    \hline

    救人指数($r_I$)    & $r_S$ & $0.087$          \\

    \hline

    倒地指数($d_I$)    & $d_S$ & $0.1$          \\

    \hline

    友伤指数($f_I$)    & $f_S$ & $0.100$          \\

    \hline

    硝石指数($n_I$)    & $n_S$ & $0.300$          \\

    \hline

    补给指数($s_I$)    & $s_S$ & $0.013$          \\

    \hline

    采集指数($m_I$)    & $m_S$ & $0.200$          \\

    \hline
\end{longtable}


\subsection{(输出型)侦察}

\begin{description}
    \item[$S'_1$] 对单输出.
    \item[$S'_2$] 处理高威胁单位.
    \item[$S'_3$] 保证硝石供应.
    \item[$S'_4$] 采矿.
\end{description}

约束条件:

\begin{description}
    \item[$S'_1$] $k_{S'} + D_{S'} + P_{S'} \ge 0.6, P_{S'} > D_{S'} > k_{S'}, k_{S'}=0$.
    \item[$S'_2$] $P_{S'} \ge 0.3$.
    \item[$S'_3$] $n_{S'} \ge 0.1$.
    \item[$S'_4$] $n_{S'} + p_{S'} \ge 0.15$.
    \item[$A_1$] $f_{S'} \ge 0.1$.
    \item[$A_2$] 我们估计,平均每局总计输出大约为70K,输出型侦察平均输出应大于15K,则不带权输出指数$D_I$应大约为$D^0 = \frac{3}{14}$,而带权输出指数$D^{S'}_I$应大于$D_I$

          我们估计,输出型侦察每局\textbf{正常}倒地次数为1.5次($d^0$),每局所有人总计倒地次数为6次($d^1$),则由$A_2$:

          $\frac{D^0}{d^0} \cdot D_{S'}> \frac{1}{d^1} \cdot d_{S'}$
          解得$D_{S'} > \frac{d^0 \cdot d_{S'}}{d^1 \cdot D^0} = \frac{7}{6} \cdot d_{S'}$
    \item[$A_3$]我们估计,输出型侦察的补给指数约为$\frac{1}{4}$($s^0$)(钻机$\frac{1}{4}$,枪手$\frac{1}{4}$,工程$\frac{3}{8}$,(辅助型)侦察$\frac{1}{8}$)

          我们估计,8K伤害对应一份补给,即一份补给对应约$\frac{8D^0}{15} = \frac{4}{35}$

          则由$A_3$:$\frac{8D^0}{15} \cdot D_{S'} > s^0 \cdot s_{S'}$

          解得$s_{S'} < \frac{8D^0 \cdot D_{S'}}{15 s^0} = \frac{16}{35} D_{S'}$.

    \item[$A_4$] $d_{S'} \ge r_{S'}$.
\end{description}


\begin{longtable}{|>{\centering\arraybackslash}p{8em}|>{\centering\arraybackslash}p{4em}|>{\centering\arraybackslash}p{6em}|}
    \hline

    项目                & 权重标识     & 参考值     \endhead

    \hline

    击杀数指数($k_I^{S'}$) & $k_{S'}$ & $0.000$          \\

    \hline

    输出指数($D_I^{S'}$)  & $D_{S'}$ & $0.250$          \\

    \hline

    高威胁目标($P_I$)      & $P_{S'}$ & $0.350$          \\

    \hline

    救人指数($r_I$)       & $r_{S'}$ & $0.055$          \\

    \hline

    倒地指数($d_I$)       & $d_{S'}$ & $0.065$          \\

    \hline

    友伤指数($f_I$)       & $f_{S'}$ & $0.100$          \\

    \hline

    硝石指数($n_I$)       & $n_{S'}$ & $0.100$          \\

    \hline

    补给指数($s_I$)       & $s_{S'}$ & $0.030$          \\

    \hline

    采集指数($m_I$)       & $m_{S'}$ & $0.050$          \\

    \hline
\end{longtable}

\section{玩家KPI}

设玩家在任务$i$中所选的角色为$c_i$,在该任务中的玩家指数为$p_i$,其$mKPI$为$a$.

则玩家总体KPI为:

\begin{equation}
    KPI = \frac{\sum_{i = 1}^{n} p_i \cdot a}{\sum_{i = 1}^{n} p_i}
\end{equation}

\appendix

\section{统计数据}
\label{sec:statistic}

以下数据仅供参考,在实际计算KPI时,将根据所有任务的信息计算下列数据及$\Gamma$.

有效局数:53,独立玩家数:66.

以下“有效数据量”为在所有有效任务中,该角色的玩家指数之和.

\begin{longtable}{|>{\centering\arraybackslash}p{3em}|>{\centering\arraybackslash}p{3em}|>{\centering\arraybackslash}p{5em}|>{\centering\arraybackslash}p{5em}|}
    \hline

    角色 & 有效数据数量 & 平均击杀数 & 修正指标$\gamma^k$ \\

    \hline

    工程 & 55.95  & 188   & 2.848          \\

    \hline

    枪手 & 34.71  & 111   & 1.682          \\

    \hline

    钻机 & 35.75  & 111   & 1.682          \\

    \hline

    侦察 & 63.87  & 66    & 1.000          \\

    \hline

    \caption{角色击杀数与修正指标}

    \label{tab:kill_by_character}
\end{longtable}

\begin{longtable}{|>{\centering\arraybackslash}p{3em}|>{\centering\arraybackslash}p{3em}|>{\centering\arraybackslash}p{5em}|>{\centering\arraybackslash}p{5em}|}
    \hline

    角色 & 有效数据数量 & 平均伤害   & 修正指标$\gamma^D$ \\

    \hline

    工程 & 55.95  & 22.50K & 2.204          \\

    \hline

    枪手 & 34.71  & 13.86K & 1.357          \\

    \hline

    钻机 & 35.75  & 11.99K & 1.174          \\

    \hline

    侦察 & 63.87  & 10.21K & 1.000          \\

    \hline

    \caption{角色输出与修正指标}

    \label{tab:damage_by_character}
\end{longtable}

\begin{longtable}{|>{\centering\arraybackslash}p{3em}|>{\centering\arraybackslash}p{3em}|>{\centering\arraybackslash}p{5em}|>{\centering\arraybackslash}p{5em}|}
    \hline

    角色 & 有效数据数量 & 平均硝石采集量 & 修正指标$\gamma^n$ \\

    \hline

    侦察 & 63.87  & 138     & 3.000          \\

    \hline

    工程 & 55.95  & 78      & 1.696          \\

    \hline

    枪手 & 34.71  & 55      & 1.196          \\

    \hline

    钻机 & 35.75  & 46      & 1.000          \\

    \hline
    \caption{角色硝石采集量与修正指标}

    \label{tab:nitra_by_character}
\end{longtable}

\begin{longtable}{|>{\centering\arraybackslash}p{3em}|>{\centering\arraybackslash}p{3em}|>{\centering\arraybackslash}p{5em}|>{\centering\arraybackslash}p{5em}|}
    \hline

    角色 & 有效数据数量 & 平均矿石采集量 & 修正指标$\gamma^m$ \\

    \hline

    侦察 & 63.87  & 256     & 2.612          \\

    \hline

    工程 & 55.95  & 135     & 1.378          \\

    \hline

    枪手 & 34.71  & 107     & 1.092          \\

    \hline

    钻机 & 35.75  & 98      & 1.000          \\

    \hline

    \caption{角色矿石采集量与修正指标}

    \label{tab:minerals_by_character}
\end{longtable}

\section{权值表}

对于权值表中未出现的敌人,在计算加权值时,按默认值计算.

高威胁目标权值表的默认值为0,角色权值表为1.

\begin{longtable}{|>{\centering\arraybackslash}p{20em}|>{\centering\arraybackslash}p{10em}|>{\centering\arraybackslash}p{3em}|}
    \hline

    ID                                     & 中文名            & 权值   \endhead

    \hline
    \verb|ED_Spider_Stalker|               & \verb|潜影异虫|    & 16.0          \\
    \hline

    \verb|ED_CaveLeech|                    & \verb|洞穴水蛭|    & 8.0           \\
    \hline

    \verb|ED_TentacleNode|                 & \verb|瓦托克鳞甲荆节| & 8.0           \\
    \hline

    \verb|ED_BarrageInfector|              & \verb|瘟疫霰射吐珠|  & 7.0           \\
    \hline

    \verb|ED_Spider_Stinger|               & \verb|蛭尾异虫|    & 7.0           \\
    \hline

    \verb|ED_TentaclePlant|                & \verb|瓦托克鳞甲荆丛| & 7.0           \\
    \hline

    \verb|ED_Grabber|                      & \verb|捕手异虫蝇|   & 6.0           \\
    \hline

    \verb|ED_ShootingPlant|                & \verb|瘟疫吐珠|    & 6.0           \\
    \hline

    \verb|ED_Spider_Lobber|                & \verb|脓毒异虫|    & 6.0           \\
    \hline

    \verb|ED_Spider_Shooter|               & \verb|吐酸异虫|    & 6.0           \\
    \hline

    \verb|ED_Mactera_Shooter_HeavyVeteran| & \verb|坚甲异虫蝇|   & 5.0           \\
    \hline

    \verb|ED_Mactera_Shooter_Normal|       & \verb|吐刺异虫蝇|   & 5.0           \\
    \hline

    \verb|ED_Mactera_TripleShooter|        & \verb|三颚异虫蝇|   & 5.0           \\
    \hline

    \verb|ED_Spider_RapidShooter|          & \verb|速射酸虫|    & 5.0           \\
    \hline

    \verb|ED_Spider_Spitter|               & \verb|吐丝异虫|    & 5.0           \\
    \hline

    \verb|ED_FacilityTurret_Burst|         & \verb|连射炮塔|    & 4.0           \\
    \hline

    \verb|ED_FacilityTurret_Sniper|        & \verb|狙击炮塔|    & 4.0           \\
    \hline

    \verb|ED_Spider_Buffer|                & \verb|异虫典狱长|   & 3.0           \\
    \hline

    \verb|ED_Bomber|                       & \verb|黏液轰炸蝇|   & 2.0           \\
    \hline

    \verb|ED_FacilityTurret_Barrier|       & \verb|推斥炮塔|    & 2.0           \\
    \hline

    \verb|ED_Spider_Exploder|              & \verb|自爆异虫|    & 1.0           \\
    \hline

    \verb|ED_Woodlouse|                    & \verb|丘罗那地虱|   & 1.0           \\
    \hline

    \verb|ED_Woodlouse_Youngling|          & \verb|地虱幼体|    & 1.0           \\
    \hline
    \caption{高威胁目标权值表}
\end{longtable}

\begin{longtable}{|>{\centering\arraybackslash}p{20em}|>{\centering\arraybackslash}p{10em}|>{\centering\arraybackslash}p{3em}|}
    \hline

    ID                               & 中文名           & 权值   \endhead

    \hline
    \verb|ED_Flea|                   & \verb|脓蚤|     & 5.0           \\
    \hline

    \verb|ED_InfestationLarva|       & \verb|肉食幼虫|   & 5.0           \\
    \hline

    \verb|ED_Spider_Grunt|           & \verb|战士异虫|   & 5.0           \\
    \hline

    \verb|ED_Spider_Grunt_Attacker|  & \verb|刀锋异虫|   & 5.0           \\
    \hline

    \verb|ED_Spider_Grunt_Guard|     & \verb|护卫异虫|   & 5.0           \\
    \hline

    \verb|ED_Spider_Hoarder|         & \verb|嗜矿异虫|   & 5.0           \\
    \hline

    \verb|ED_Spider_Spawn|           & \verb|异虫幼虫|   & 5.0           \\
    \hline

    \verb|ED_Spider_Swarmer|         & \verb|蜂拥异虫|   & 5.0           \\
    \hline

    \verb|ED_PatrolBot|              & \verb|巡逻机器人|  & 3.0           \\
    \hline

    \verb|ED_Shredder|               & \verb|粉碎者|    & 3.0           \\
    \hline

    \verb|ED_Spider_Boss_Heavy|      & \verb|巢主无畏异虫| & 3.0           \\
    \hline

    \verb|ED_Spider_Boss_TwinA|      & \verb|强弩无畏异虫| & 3.0           \\
    \hline

    \verb|ED_Spider_Boss_TwinB|      & \verb|重斧无畏异虫| & 3.0           \\
    \hline

    \verb|ED_Spider_ShieldTank|      & \verb|暴君异虫|   & 3.0           \\
    \hline

    \verb|ED_Spider_Tank_Boss|       & \verb|无畏异虫|   & 3.0           \\
    \hline

    \verb|ED_Spider_Tank_HeavySpawn| & \verb|哨卫异虫|   & 3.0           \\
    \hline

    \verb|ED_Spider_Tank|            & \verb|禁卫异虫|   & 2.0           \\
    \hline
    \caption{钻机权值表}
\end{longtable}

\begin{longtable}{|>{\centering\arraybackslash}p{20em}|>{\centering\arraybackslash}p{10em}|>{\centering\arraybackslash}p{3em}|}
    \hline

    ID                                 & 中文名           & 权值   \endhead

    \hline
    \verb|ED_Flea|                     & \verb|脓蚤|     & 5.0           \\
    \hline

    \verb|ED_Spider_Hoarder|           & \verb|嗜矿异虫|   & 5.0           \\
    \hline

    \verb|ED_Spider_Grunt_Attacker|    & \verb|刀锋异虫|   & 4.0           \\
    \hline

    \verb|ED_Spider_Tank_HeavySpawn|   & \verb|哨卫异虫|   & 4.0           \\
    \hline

    \verb|ED_Spider_Boss_Heavy|        & \verb|巢主无畏异虫| & 3.0           \\
    \hline

    \verb|ED_Spider_Boss_TwinA|        & \verb|强弩无畏异虫| & 3.0           \\
    \hline

    \verb|ED_Spider_Boss_TwinB|        & \verb|重斧无畏异虫| & 3.0           \\
    \hline

    \verb|ED_Spider_ExploderTank|      & \verb|大自爆虫|   & 3.0           \\
    \hline

    \verb|ED_Spider_ExploderTank_King| & \verb|自爆王虫|   & 3.0           \\
    \hline

    \verb|ED_Spider_Grunt|             & \verb|战士异虫|   & 3.0           \\
    \hline

    \verb|ED_Spider_Grunt_Guard|       & \verb|护卫异虫|   & 3.0           \\
    \hline

    \verb|ED_Spider_Tank|              & \verb|禁卫异虫|   & 3.0           \\
    \hline

    \verb|ED_Spider_Tank_Boss|         & \verb|无畏异虫|   & 3.0           \\
    \hline

    \verb|ED_FlyingSmartRock|          & \verb|飞石|     & 2.0           \\
    \hline

    \verb|ED_InfestationLarva|         & \verb|肉食幼虫|   & 2.0           \\
    \hline

    \verb|ED_PatrolBot|                & \verb|巡逻机器人|  & 2.0           \\
    \hline

    \verb|ED_Shredder|                 & \verb|粉碎者|    & 2.0           \\
    \hline

    \verb|ED_Spider_ShieldTank|        & \verb|暴君异虫|   & 2.0           \\
    \hline

    \verb|ED_Spider_Spawn|             & \verb|异虫幼虫|   & 2.0           \\
    \hline

    \verb|ED_Spider_Swarmer|           & \verb|蜂拥异虫|   & 2.0           \\
    \hline
    \caption{工程权值表}
\end{longtable}

\begin{longtable}{|>{\centering\arraybackslash}p{20em}|>{\centering\arraybackslash}p{10em}|>{\centering\arraybackslash}p{3em}|}
    \hline

    ID                                 & 中文名           & 权值   \endhead

    \hline
    \verb|ED_Flea|                     & \verb|脓蚤|     & 5.0           \\
    \hline

    \verb|ED_Spider_Hoarder|           & \verb|嗜矿异虫|   & 5.0           \\
    \hline

    \verb|ED_FlyingSmartRock|          & \verb|飞石|     & 4.0           \\
    \hline

    \verb|ED_Spider_Boss_Heavy|        & \verb|巢主无畏异虫| & 4.0           \\
    \hline

    \verb|ED_Spider_Boss_TwinA|        & \verb|强弩无畏异虫| & 4.0           \\
    \hline

    \verb|ED_Spider_Boss_TwinB|        & \verb|重斧无畏异虫| & 4.0           \\
    \hline

    \verb|ED_Spider_ExploderTank|      & \verb|大自爆虫|   & 4.0           \\
    \hline

    \verb|ED_Spider_ExploderTank_King| & \verb|自爆王虫|   & 4.0           \\
    \hline

    \verb|ED_Spider_Grunt_Attacker|    & \verb|刀锋异虫|   & 4.0           \\
    \hline

    \verb|ED_Spider_Tank_Boss|         & \verb|无畏异虫|   & 4.0           \\
    \hline

    \verb|ED_Spider_Tank_HeavySpawn|   & \verb|哨卫异虫|   & 4.0           \\
    \hline

    \verb|ED_Spider_Grunt|             & \verb|战士异虫|   & 3.0           \\
    \hline

    \verb|ED_Spider_Grunt_Guard|       & \verb|护卫异虫|   & 3.0           \\
    \hline

    \verb|ED_Spider_ShieldTank|        & \verb|暴君异虫|   & 3.0           \\
    \hline

    \verb|ED_Spider_Tank|              & \verb|禁卫异虫|   & 3.0           \\
    \hline

    \verb|ED_FacilityCaretaker|        & \verb|看守者|    & 2.0           \\
    \hline

    \verb|ED_InfectedMule|             & \verb|BET-C|  & 2.0           \\
    \hline

    \verb|ED_InfestationLarva|         & \verb|肉食幼虫|   & 2.0           \\
    \hline

    \verb|ED_PatrolBot|                & \verb|巡逻机器人|  & 2.0           \\
    \hline

    \verb|ED_Shredder|                 & \verb|粉碎者|    & 2.0           \\
    \hline

    \verb|ED_Spider_Spawn|             & \verb|异虫幼虫|   & 2.0           \\
    \hline

    \verb|ED_Spider_Swarmer|           & \verb|蜂拥异虫|   & 2.0           \\
    \hline
    \caption{枪手权值表}
\end{longtable}

\begin{longtable}{|>{\centering\arraybackslash}p{20em}|>{\centering\arraybackslash}p{10em}|>{\centering\arraybackslash}p{3em}|}
    \hline

    ID                               & 中文名             & 权值   \endhead

    \hline
    \verb|ED_Flea|                   & \verb|脓蚤|       & 5.0           \\
    \hline

    \verb|ED_Spider_Hoarder|         & \verb|嗜矿异虫|     & 5.0           \\
    \hline

    \verb|ED_FlyingSmartRock|        & \verb|飞石|       & 3.0           \\
    \hline

    \verb|ED_InfectedMule|           & \verb|BET-C|    & 3.0           \\
    \hline

    \verb|ED_JellyBreeder|           & \verb|纳多赛特饲育水母| & 3.0           \\
    \hline

    \verb|ED_FacilityCaretaker|      & \verb|看守者|      & 2.0           \\
    \hline

    \verb|ED_Spider_Boss_Heavy|      & \verb|巢主无畏异虫|   & 2.0           \\
    \hline

    \verb|ED_Spider_Boss_TwinA|      & \verb|强弩无畏异虫|   & 2.0           \\
    \hline

    \verb|ED_Spider_Boss_TwinB|      & \verb|重斧无畏异虫|   & 2.0           \\
    \hline

    \verb|ED_Spider_Tank_Boss|       & \verb|无畏异虫|     & 2.0           \\
    \hline

    \verb|ED_Spider_Tank_HeavySpawn| & \verb|哨卫异虫|     & 2.0           \\
    \hline
    \caption{辅助型侦察权值表}
\end{longtable}

\begin{longtable}{|>{\centering\arraybackslash}p{20em}|>{\centering\arraybackslash}p{10em}|>{\centering\arraybackslash}p{3em}|}
    \hline

    ID                                 & 中文名             & 权值   \endhead

    \hline
    \verb|ED_Flea|                     & \verb|脓蚤|       & 5.0           \\
    \hline

    \verb|ED_Spider_Hoarder|           & \verb|嗜矿异虫|     & 5.0           \\
    \hline

    \verb|ED_InfectedMule|             & \verb|BET-C|    & 4.0           \\
    \hline

    \verb|ED_Spider_Tank_HeavySpawn|   & \verb|哨卫异虫|     & 4.0           \\
    \hline

    \verb|ED_FacilityCaretaker|        & \verb|看守者|      & 3.0           \\
    \hline

    \verb|ED_JellyBreeder|             & \verb|纳多赛特饲育水母| & 3.0           \\
    \hline

    \verb|ED_Spider_Boss_Heavy|        & \verb|巢主无畏异虫|   & 3.0           \\
    \hline

    \verb|ED_Spider_Boss_TwinA|        & \verb|强弩无畏异虫|   & 3.0           \\
    \hline

    \verb|ED_Spider_Boss_TwinB|        & \verb|重斧无畏异虫|   & 3.0           \\
    \hline

    \verb|ED_Spider_ExploderTank|      & \verb|大自爆虫|     & 3.0           \\
    \hline

    \verb|ED_Spider_ExploderTank_King| & \verb|自爆王虫|     & 3.0           \\
    \hline

    \verb|ED_Spider_ShieldTank|        & \verb|暴君异虫|     & 3.0           \\
    \hline

    \verb|ED_Spider_Tank|              & \verb|禁卫异虫|     & 3.0           \\
    \hline

    \verb|ED_Spider_Tank_Boss|         & \verb|无畏异虫|     & 3.0           \\
    \hline

    \verb|ED_FlyingSmartRock|          & \verb|飞石|       & 2.0           \\
    \hline

    \verb|ED_PatrolBot|                & \verb|巡逻机器人|    & 2.0           \\
    \hline
    \caption{输出型侦察权值表}
\end{longtable}
\begin{longtable}{|>{\centering\arraybackslash}p{20em}|>{\centering\arraybackslash}p{10em}|>{\centering\arraybackslash}p{3em}|}
    \hline 
    
    ID                                     & 中文名            & 权值   \endhead
    
    \hline
            \verb|RES_COLLECT_Barley1| & \verb|大麦球果| & 64 \\ 
     \hline 
    
            \verb|RES_CARVED_Bismor| & \verb|蜂母石| & 4 \\ 
     \hline 
    
            \verb|RES_CARVED_Magnite| & \verb|吸铁石| & 4 \\ 
     \hline 
    
            \verb|RES_CARVED_Phazyonite| & \verb|方晶辉石| & 4 \\ 
     \hline 
    
            \verb|RES_CARVED_Umanite| & \verb|乌玛石| & 4 \\ 
     \hline 
    
            \verb|RES_VEIN_Croppa| & \verb|铜矿| & 4 \\ 
     \hline 
    
            \verb|RES_CARVED_Hollomite| & \verb|容和石| & 2 \\ 
     \hline 
    
            \verb|RES_COLLECT_Apoca_Bloom| & \verb|彼岸花| & 2 \\ 
     \hline 
    
            \verb|RES_COLLECT_Boolo| & \verb|布洛蘑菇| & 2 \\ 
     \hline 
    
            \verb|RES_COLLECT_Ebonut| & \verb|硬胶果实| & 2 \\ 
     \hline 
    
            \verb|RES_COLLECT_Fossil| & \verb|化石| & 2 \\ 
     \hline 
    
            \verb|RES_VEIN_Dystrum| & \verb|异镝| & 2 \\ 
     \hline 
    
            \verb|RES_VEIN_Morkite| & \verb|墨菱石| & 2 \\ 
     \hline 
    
            \verb|RES_CARVED_OilShale| & \verb|油页岩| & 1 \\ 
     \hline 
    
            \verb|RES_COLLECT_Barley2| & \verb|酵母松果| & 1 \\ 
     \hline 
    
            \verb|RES_COLLECT_Barley3| & \verb|麦芽星果| & 1 \\ 
     \hline 
    
            \verb|RES_COLLECT_Barley4| & \verb|淀粉坚果| & 1 \\ 
     \hline 
    
            \verb|RES_COLLECT_MorkiteSeed| & \verb|墨菱种子| & 1 \\ 
     \hline 
    
            \verb|RES_EMBED_Enor| & \verb|妙绝珠| & 1 \\ 
     \hline 
    
            \verb|RES_EMBED_Jadiz| & \verb|玉石| & 1 \\ 
     \hline 
    
            \verb|RES_VEIN_Gold| & \verb|黄金| & 1 \\ 
     \hline 
    
            \verb|RES_VEIN_Nitra| & \verb|硝石| & 1 \\ 
     \hline 
    \caption{矿物采集权值表}
    \end{longtable}

\begin{thebibliography}{99}
    \bibitem{bilibili-scout} 猫猫爱吃875小饼干.【深岩银河】从零开始的顶侦培养计划---第一章基础介绍(入门篇). https://www.bilibili.com/video/BV1ig4y197GB.
    \bibitem{tieba-scout} 苏特施季里茨. 深岩银河侦察职业定位和武器选择. https://tieba.baidu.com/p/7819452549.
    \bibitem{tieba-all} 红莲paduma. 给萌新的一些话. https://tieba.baidu.com/p/7253319800.
    \bibitem{xiaoheihe-all} 寻妳足迹. 深岩银河从入门到入坑的一条龙指南. https://api.xiaoheihe.cn/v3/bbs/app/api/web/share?link\_id=119582418.
\end{thebibliography}

\end{document}